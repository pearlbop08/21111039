\documentclass[11pt]{article}
\title{\textbf{An Introduction To Human Body}}
\usepackage{graphicx}
\graphicspath{{image/}}
\begin{document}
\maketitle
\centering
\emph{\large Submitted by:}
\large Priyanka Harde

\large Roll No.:21111039
\begin{figure}[h]
\begin{center}
\includegraphics[scale=0.6]{nit.jpg}
\end{center}
\end{figure}

\large Under the supervision of : SAURABH GUPTA 
\clearpage 
\tableofcontents
\clearpage
\section{Anatomy and Physiology}

\raggedright Anatomy (a-NAT-oˉ-mē; ana- = up; -tomy = process of cutting) is the
science of body structures and the relationships among them.Whereas anatomy deals with structures of the body, physiology (fiz′-
ē-OL-oˉ-jē; physio- = nature; -logy = study of) is the science of body
functions—how the body parts. 
\subsection{Levels of Structural
Organization and
Body Systems}

There are six levels of structural organization and 11 systems of human body.Chemical,Cellular,Tissue,Organ,system,organismal level and integumentary ,skeletal,muscular,nervous,endocrine,cardiovascular,lymphatic,respiratory,digestive, 
urinary,reproductive systems.
\subsubsection{Basic Life Processes}

Metabolism,Responsivness,Movement,Growth,Differentiation,Reproduction.
\section{Homeostasis}

Homeostasis (hoˉ′-mē-oˉ-STA - -sis; homeo- = sameness; -stasis = standing
still) is the maintenance of relatively stable conditions in the
body’s internal environment. It occurs because of the ceaseless interplay
of the body’s many regulatory systems. Homeostasis is a dynamic
condition. In response to changing conditions, the body’s parameters
can shift among points in a narrow range that is compatible with
maintaining life. For example, the level of glucose in blood normally
stays between 70 and 110 milligrams of glucose per 100 milliliters of
blood.An important aspect of homeostasis is maintaining the volume and
composition of body fluids, dilute, watery solutions containing dissolved
chemicals that are found inside cells as well as surrounding
them.
\subsection{Control of Homeostasis}
In most cases the disruption of
homeostasis is mild and temporary, and the responses of body cells
quickly restore balance in the internal environment. However, in
some cases the disruption of homeostasis may be intense and prolonged,
as in poisoning, overexposure to temperature extremes,
severe infection, or major surgery.
In most cases the disruption of
homeostasis is mild and temporary, and the responses of body cells
quickly restore balance in the internal environment. However, in
some cases the disruption of homeostasis may be intense and prolonged,
as in poisoning, overexposure to temperature extremes,
severe infection, or major surgery.


\textsc{Feedback Systems}
Receptor is a body structure that monitors changes in a controlled
condition and sends input to a control center,Control centre evaluates the input it receives from receptors, and
generates output commands when they are needed. Output from
the control center typically occurs as nerve impulses, or hormones. ,Effector is a body structure that receives output
from the control center and produces a response or effect that
changes the controlled condition.


\textsc{NEGATIVE FEEDBACK SYSTEMS}
A negative feedback system reverses a change in a controlled condition.\textsc{POSITIVE FEEDBACK SYSTEMS}
Unlike a negative feedback system, a
positive feedback system tends to strengthen or reinforce a change
in one of the body’s controlled conditions.
\section{Body Cavities}
Body cavities are spaces that enclose internal organs. Bones, muscles,
ligaments, and other structures separate the various body cavities from
one another.1)Cranial cavity-Formed by cranial bones and contains brain.2)Vertebral canal-Formed by vertebral column and contains spinal
cord and the beginnings of spinal nerves.3)Thoracic cavity-Chest cavity; contains pleural and pericardial
cavities and the mediastinum.4)Pleural cavity-A potential space between the layers of the
pleura that surrounds a lung.5)Pericardial cavity-A potential space between the layers of the
pericardium that surrounds the heart.6)Mediastinum-Central portion of thoracic cavity between the
lungs; extends from sternum to vertebral column
and from first rib to diaphragm; contains heart,
thymus, esophagus, trachea, and several large
blood vessels.7)Abdominal cavity-Contains stomach, spleen, liver, gallbladder,
small intestine, and most of large intestine; the
serous membrane of the abdominal cavity is
the peritoneum.8)Pelvic cavity-Contains urinary bladder, portions of large
intestine, and internal organs of reproduction.
\subsubsection{Aging and Homeostasis}
Aging is a normal process characterized by a
progressive decline in the body’s ability to restore homeostasis.
Aging produces observable changes in structure and function and
increases vulnerability to stress and disease.
\section{Medical Imaging}
Medical imaging refers to techniques and procedures used to create
images of the human body. Various types of medical imaging allow
visualization of structures inside our bodies and are increasingly
helpful for precise diagnosis of a wide range of anatomical and physiological  disorders.
eg- 1)Radiography-Relatively inexpensive, quick, and simple to perform; usually
provides sufficient information for diagnosis. X-rays do not easily pass
through dense structures, so bones appear white. Hollow structures,
such as the lungs, appear black. Structures of intermediate density, such
as skin, fat, and muscle, appear as varying shades of gray. At low doses,
x-rays are useful for examining soft tissues such as the breast.
\paragraph{2})MAGNETIC RESONANCE IMAGING (MRI)-The body is exposed to a high-energy magnetic field, which
causes protons (small positive particles within atoms, such as hydrogen)
in body fluids and tissues to arrange themselves in relation to the field.
Relatively safe but cannot be used on patients with metal
in their bodies. Shows fine details for soft tissues but not for bones.
Most useful for differentiating between normal and abnormal tissues.
\paragraph{3})COMPUTED TOMOGRAPHY (CT)-an x-ray
beam traces an arc at multiple angles around a section of the body.Visualizes soft tissues and organs with much more detail
than conventional radiographs. Differing tissue densities show up as
various shades of gray.
\paragraph{4})ULTRASOUND SCANNING-High-frequency sound waves produced by a handheld wand reflect
off body tissues and are detected by the same instrument.
\paragraph{5})CORONARY (CARDIAC) COMPUTED TOMOGRAPHY
ANGIOGRAPHY (CCTA) SCAN-An iodine containing
contrast medium is injected into a vein and a beta blocker is
given to decrease heart rate. Then, numerous x-ray beams trace an arc
around the heart and a scanner detects the x-ray beams and transmits
them to a computer.Used primarily to determine if there are any coronary artery blockages.
\paragraph{6})POSITRON EMISSION TOMOGRAPHY (PET)-
A substance that emits positrons (positively charged particles)
is injected into the body, where it is taken up by tissues. The collision of
positrons with negatively charged electrons in body tissues produces
gamma rays (similar to x-rays) that are detected by gamma cameras.Used to study the physiology of body structures, such as
metabolism in the brain or heart.
\paragraph{7})ENDOSCOPY-Endoscopy involves the visual examination of the inside of body
organs or cavities using a lighted instrument with lenses called an endoscope.
\paragraph{8})RADIONUCLIDE SCANNING-A radionuclide (radioactive substance) is
introduced intravenously into the body and carried by the
blood to the tissue to be imaged. Gamma rays emitted by the
radionuclide are detected by a gamma camera outside the
subject, and the data are fed into a computer.Used to study activity of a tissue or organ, such
as searching for malignant tumors in body tissue or scars that
may interfere with heart muscle activity.
\end{document}
