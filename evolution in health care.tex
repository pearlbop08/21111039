\documentclass[11pt]{article}
\title{\Huge \textbf{ASSIGNMENT}}
\usepackage{graphicx}
\graphicspath{{image/}}
\begin{document}
\maketitle
\centering On

\Huge "Evolution of Modern Health care"
\setlength{\parskip}{0.5em}

\emph{\large Submitted by:}
\large Priyanka Harde
\large Roll No.:21111039
\begin{figure}[h]
\begin{center}
\includegraphics[scale=0.7]{nit.jpg}
\end{center}
\end{figure}

\textsc{\Large NATIONAL INSTITUTE OF TECHNOLOGY}

\large Under the supervision of : SAURABH GUPTA 
\clearpage 
\tableofcontents
\clearpage
\section{\huge  ACKNOWLEDGEMENT}
\Large \raggedright I would like to express my sincere thanks and gratitude to Saurabh Sir for letting me work on this project. I am very grateful to him for his support and guidance in completing this project.

I am thankful to my parents as well. I was able to successfully complete this project with the help of their guidance and support. Finally, I want to thank all my dear friends as well..

\section{\textbf{\huge INTRODUCTION}}
\raggedright \Large A health system, also known as health care system or healthcare system, is the organization of people, institutions, and resources that deliver health care services to meet the health needs of target populations.

The Indian Medical Services was formed in 1896 and the subsequent transfer of public health, sanitation, and vital statistics to the provinces took place in 1919. A new department to cater to education and health was constituted in 1912, with public health physicians in medical colleges entrusted with teaching hygiene.
\section{\huge Public Health in British India}
The British Imperial government set up and strengthened an organized medical system in Colonial India that replaced the indigenous Indian and Arabic medicine systems. Slow progress in early years was due to indifference on the part of people and a lack of funds and medical professionals on the part of the government.
\section{\huge 20th Century Public Health}
This are the following reasons why our health sector has improved:
\begin{itemize}
\item \large Vaccination to reduce epidemic diseases
\item \large Improved motor vehicle safety
\item \large Safer workplaces
\item \large Control of infectious diseases
\item \large Decline in death from cardiovascular disease
\item \large Food Safety
\item \large Improvements in maternal and child health 
\item \large Family planning
\item \large Fluoridation of drinking water
\item \large  Reductions in prevalence of tobacco use
\end{itemize}
\clearpage
\section{Today}
\Large India, the land of Ayurveda, has a wide variety of special treatments to offer. In addition, there are hospitals practicing modern medicine that provide quality service at an affordable cost. When compared to the expense of medical treatment in Western countries, India’s facilities for treatment, natural beauty and tourist destinations across the country will make it a popular destination for people of all nationalities seeking health care.
\begin{figure}[h]
\includegraphics[scale=0.4]{pyramid of healthcare.png}
\end{figure}


In the year 2006, the quality council of India, through the National Accreditation Board for Hospitals (NABH) has come out with hospital standards that are applicable to Indian hospitals. The likelihood of an insurance boom in the health care sector and the potential for health tourism are important reasons for accrediting the hospitals. Therefore, accreditation and quality health service will be the main agenda of hospitals in the years to come.

\subsection{Conclusion}

India has made striking progress in health standards in the post-independence era. Still, many feel that the budgetary resources for the health sector should be increased. International developments in information technology need to be utilized at the national level in an attempt for health data documentation. The sustained efforts to control the country’s population and the political will to march towards the millennium development goals in health will help India to make a significant impact in the international health scene.
  
  
  
\subsection{References :}  \Large healthmanagement.org ,www.ncbi.nlm.nih.gov

 







\end{document}


