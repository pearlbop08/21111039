\documentclass[11pt]{article}
\title{\Huge \textbf{ASSIGNMENT}}
\usepackage{graphicx}

\graphicspath{{image/}}

\begin{document}
\maketitle
\begin{center}
\huge On 
\end{center} 

\huge\centering 5 Medical Devices

\setlength{\parskip}{0.5em}\emph{\large Submitted by: }
\large Priyanka Harde

\large Roll No.:21111039
\begin{figure}[h]
\begin{center}
\includegraphics[scale=0.5]{nit.jpg}
\end{center}

\end{figure}



\textsc{\large NATIONAL INSTITUTE OF TECHNOLOGY ,RAIPUR}
\setlength{\parskip}{0.5em}




\textit{\Large Under the Supervision of:}
\textbf{\Large Saurabh Gupta}
\clearpage
\tableofcontents

\raggedright
\section{\Huge \raggedright ACKNOWLEDGEMENT}
\Large I would like to express my sincere thanks and gratitude to Saurabh Sir for letting me work on this project. I am very grateful to him for his support and guidance in completing this project.

I am thankful to my parents as well. I was able to successfully complete this project with the help of their guidance and support. Finally, I want to thank all my dear friends as well..
\section{\textbf{\Huge INTRODUCTION}}

\raggedright
\Large A medical device is an instrument ,implant or other similar o related article , which is intended for use in the diagnosis of disease or other condition,or in the cure, treatment or prevention of diseases or intended to affect the structure or any function of the body and which does not achieve any of its primary intended purposes through its chemical action within or on the body. 


\setlength{\parskip}{0.5em}
\raggedright
\textsf{THE 5 MEDICAL DEVICES ARE}:
\begin{itemize}
\item Lab Incubator
\item Mobile MRI
 \item Needle Free Injector
 \item Oxygenator
 \item Therapeutic Intravascular Ultrasound
\end{itemize}


\begin{center}
\section{\textbf{\Huge Lab Incubator}}
\end{center}
\subsection{What is Incubator?}

\begin{figure}[h]
\centering
\includegraphics[scale=0.7]{1.jpg}
\caption{Interior of a CO2 incubator used in cell culture}
\end{figure}

\setlength{\parskip}{0.01em}
\raggedright
An incubator is a device used to grow and maintain microbiological cultures or cell cultures. The incubator maintains optimal temperature, humidity and other conditions such as the CO2 and oxygen content of the atmosphere inside. Incubators are essential for much experimental work in cell biology, microbiology and molecular biology and are used to culture both bacterial and eukaryotic cells.
\begin{figure}[h]
\centering
\includegraphics[scale=1]{Bacteriological_incubator.jpg}
\caption{A Bacteriological incubator}
\end{figure}
\raggedright
\subsection{History of the laboratory incubator}
\raggedright From aiding in hatching chicken eggs to enabling scientists to understand and develop vaccines for deadly viruses, the laboratory incubator has seen numerous applications over the years it has been in use. The incubator has also provided a foundation for medical advances and experimental work in cellular and molecular biology.

\begin{figure}[h]
\centering
\includegraphics[scale=1.5]{Egyptian Egg oven.jpg}
\caption{Egyptian Egg oven}
\end{figure}
\raggedright An incubator is made up of a chamber with a regulated temperature. Some incubators also regulate humidity, gas composition, or ventilation within that chamber. While many technological advances have occurred since the primitive incubators first used in ancient Egypt and China, the main purpose of the incubator has remained unchanged: to create a stable, controlled environment conducive to research, study, and cultivation.
\raggedright
\subsection{The earliest incubators}
\raggedright The earliest incubators were found thousands of years ago in ancient Egypt and China, where they were used to keep chicken eggs warm. Use of incubators
\begin{figure}[h]
\centering
\includegraphics[scale=1]{Method for incubation of poultry eggs.jpg}
\caption{Method for incubation of poultry eggs}
\end{figure} 
\raggedright revolutionized food production, as it allowed chicks to hatch from eggs without requiring that a hen sit on them, thus freeing the hens to lay more eggs in a shorter period of time. Both early Egyptian and Chinese incubators were essentially large rooms that were heated by fires, where attendants turned the eggs at regular intervals to ensure even heat distribution.
\raggedright
\subsection{In the 16th and 17th century}
The incubator received an update in the 16th century when Jean Baptiste Porta drew on ancient Egyptian design to create a more modern egg incubator.
\begin{figure}[h]


 \raggedright \Large While he eventually had to discontinue his work due to the Spanish Inquisition, Rene-Antoine Ferchault de Reaumur took up the challenge in the middle of the 17th century. Reaumur warmed his incubator with a wood stove and monitored its temperature using the Reaumur thermometer, another of his inventions.
 
 \centering
\includegraphics[scale=2]{Old Réaumur scale thermometer.jpg}
\caption{Old Réaumur scale thermometer}
\end{figure}
\raggedright
\subsection{In the 19th century}
\raggedright In the 19th century, researchers finally began to recognize that the use of incubators could contribute to medical advancements. They began to experiment to find the ideal environment for maintaining cell culture stocks. These early incubators were simply made up of bell jars that contained a single lit candle. Cultures were placed near the flame on the underside of the jar's lid, and the entire jar was placed in a dry, heated oven.
\begin{figure}[h]
\centering
\includegraphics[scale=1]{Incubator invented by hess.jpg}
\caption{Incubator invented by hess}
\end{figure}

\raggedright Incubator invented by Hess
In the late 19th century, doctors realized another practical use for incubators: keeping premature or weak infants alive. The first infant incubator, used at a women's hospital in Paris, was heated by kerosene lamps. Fifty years later, Julius H. Hess, an American physician often considered to be the father of neonatology, designed an electric infant incubator that closely resembles the infant incubators in use today.
\raggedright
\subsection{In the 20th century}
\raggedright The next innovation in incubator technology came in the 1960s, when the CO2 incubator was introduced to the market. Demand came when doctors realized that they could use CO2 incubators to identify and study pathogens found in patients' bodily fluids. To do this, a sample was harvested and placed onto a sterile dish and into the incubator. The air in the incubator was kept at 37 degrees Celsius, the same temperature as the human body, and the incubator maintained the atmospheric carbon dioxide and nitrogen levels necessary to promote cell growth.

\raggedright At this time, incubators also began to be used in genetic engineering. Scientists could create biologically essential proteins, such as insulin, with the use of incubators. Genetic modification could now take place on a molecular level, helping to improve the nutritional content and resistance to pestilence and disease of fruits and vegetables.
\raggedright
\subsection{Today}
\raggedright Incubators serve a variety of functions in a scientific lab. Incubators generally maintain a constant temperature, however additional features are often built in. Many incubators also control humidity. Shaking incubators incorporate movement to mix cultures. Gas incubators regulate the internal gas composition.

\begin{figure}[h]
\centering
\includegraphics[scale=1]{Shaking incubator for culture tubes.jpg}
\caption{Shaking incubator for culture tubes}
\end{figure}
\raggedright Some incubators have a means of circulating the air inside of them to ensure even distribution of temperatures. Many incubators built for laboratory use have a redundant power source, to ensure that power outages do not disrupt experiments. Incubators are made in a variety of sizes, from tabletop models, to warm rooms, which serve as incubators for large numbers of samples.



          \setlength{\parskip}{5em}
            \centering
                ********

 \setlength{\parskip}{0.5em}
\section{\Huge Mobile MRI}
\subsection{What is MRI?}
\Large Magnetic resonance imaging (MRI) is a medical imaging technique used in radiology to form pictures of the anatomy and the physiological processes of the body. MRI scanners use strong magnetic fields, magnetic field gradients, and radio waves to generate images of the organs in the body. MRI does not involve X-rays or the use of ionizing radiation, which distinguishes it from CT and PET scans. MRI is a medical application of nuclear magnetic resonance (NMR) which can also be used for imaging in other NMR applications, such as NMR spectroscopy.
\raggedright
\subsection{ What is Mobile MRI or Portable MRI?}



\raggedright \Large Portable magnetic resonance imaging (MRI) is referred to the imaging provided by an MRI scanner that has mobility and portability. It provides MR imaging to the patient in-time and on-site, for example, in Intensive care unit (ICU) where there is danger associated with moving the patient, in an ambulance, after a disaster rescue, or in a field hospital/medical tent.
\begin{figure}
\centering
\includegraphics[scale=0.5]{moble mri scanner.jpg}
\caption{Mobile MRI}
\end{figure}
\subsection{\Large Types of MRI}
1. \large Superconducting-magnet-based portable MRI

2. \large Electromagnet-based portable

3. \large Permanent-magnet-based portable MRI




\setlength{\parskip}{5em}
            \centering
                ********

 \setlength{\parskip}{1em}
\section{\Huge Needle Free Injector}
\Large \subsection{INTRODUCTION}
\Large \raggedright Needle free injection technology (NFIT)is an extremely broad concept which include a wide range of drug delivery systems that drive drugs through the skin using any of the forces as Lorentz, Shock waves, pressure by gas or electrophoresis which propels the drug through the skin, virtually nullifying the use of hypodermic needle.

\begin{figure}[h]
\centering
\includegraphics[scale=0.6]{ needle free injector.jpg}
\caption{Needle free injector}
\end{figure}
\subsection{How it is useful}
 \raggedright This technology is not only touted to be beneficial for the pharma industry but developing world too find it highly useful in mass immunization programmes, bypassing the chances of needle stick injuries and avoiding other complications including those arising due to multiple use of single needle. The NFIT devices can be classified based on their working, type of load, mechanism of drug delivery and site of delivery. To administer a stable, safe and an effective dose through NFIT, the sterility, shelf life and viscosity of drug are the main components which should be taken care of. Technically superior needle-free injection systems are able to administer highly viscous drug products which cannot be administered by traditional needle and syringe systems, further adding to the usefulness of the technology.
 
\begin{figure}[h]
\centering
\includegraphics[scale=1]{ nfi.jpg}
\caption{Normal injection versus Needle free injection}
\end{figure}
\raggedright
NFIT devices can be manufactured in a variety of ways; however the widely employed procedure to manufacture it is by injection molding technique. There are many variants of this technology which are being marketed, such as Bioject® ZetaJetTM, Vitajet 3, Tev-Tropin® and so on. Larger investment has been made in developing this technology with several devices already being available in the market post FDA clearance and a great market worldwide.
This technology is being backed by organizations as World Health Organization, Centers for Disease Control and Prevention and various groups including Bill and Melinda Gates Foundation. This technology is not only touted to be beneficial for the pharma industry but developing world too find it highly useful in mass immunization programs, bypassing the chances of needle stick injuries and avoiding other complications including those arising due to multiple uses of single needle.Better patient compliance has been observed.


\section{\Huge Oxygenator}
\subsection{what is oxygenator?}
\Large An oxygenator is a medical device that is capable of exchanging oxygen and carbon dioxide in the blood of human patient during surgical procedures that may necessitate the interruption or cessation of blood flow in the body, a critical organ or great blood vessel. These organs can be the heart, lungs or liver, while the great vessels can be the aorta, pulmonary artery, pulmonary veins or vena cava.
\subsection{How it is useful?}
\Large \raggedright An oxygenator is typically utilized by a perfusionist in cardiac surgery in conjunction with the heart-lung machine. However, oxygenators can also be utilized in extracorporeal membrane oxygenation in neonatal intensive care units by nurses. For most cardiac operations such as coronary artery bypass grafting, the cardiopulmonary bypass is performed using a heart-lung machine (or cardiopulmonary bypass machine). The heart-lung machine serves to replace the work of the heart during the open bypass surgery. The machine replaces both the heart's pumping action and the lungs' gas exchange function. Since the heart is stopped during the operation, this permits the surgeon to operate on a bloodless, stationary heart.
\begin{figure}[h]

\includegraphics[scale=0.8]{ oxygenator 2.jpg}
\centering

\caption   
oxygenator
\end{figure}

\raggedright One component of the heart-lung machine is the oxygenator. The oxygenator component serves as the lung, and is designed to expose the blood to oxygen and remove carbon dioxide. It is disposable and contains about 2–4 m² of a membrane permeable to gas but impermeable to blood, in the form of hollow fibers. Blood flows on the outside of the hollow fibers, while oxygen flows in the opposite direction on the inside of the fibers. As the blood passes through the oxygenator, the blood comes into intimate contact with the fine surfaces of the device itself. Gas containing oxygen and medical air is delivered to the interface between the blood and the device, permitting the blood cells to absorb oxygen molecules directly.
\begin{figure}
\centering
\includegraphics[scale=0.5]{block diagram.jpg}
\caption{Block diagram}
\end{figure}



\setlength{\parskip}{5em}
            \centering
                ********

 \setlength{\parskip}{1em}
\section{\Huge \raggedright Intravascular Ultrasound}
\subsection{What is Intravascular ultrasound?}
\Large \raggedright Intravascular Ultrasound (IVUS) is a catheter-based diagnostic procedure used to view the inside of a coronary artery, providing a real-time view. IVUS shows the degree of narrowing or thickening (stenosis) of an artery by providing a visual image of the inside of the artery (the lumen) and the atheroma (membrane/cholesterol loaded white blood cells) that are hidden within the artery wall.
\subsection{Advantages of Intravascular Ultrasound}
\Large \raggedright IVUS enables a physician to get inside the artery with a camera-like device. IVUS can quantify the percentage of narrowing and give insight into the nature of the plaque. It also may reveal what in the past has been referred to as "re-stenosis" (a recurrence of the plaque buildup that may have previously been removed). There is evidence that this is not a re-stenosis but rather the IVUS's ability to see buildup that may have been missed during an angiogram and angioplasty.
\begin{figure}[h]

\includegraphics[scale=0.7]{Ultrasound.jpg}
\centering

\caption{Intravascular Ultrasound}
\end{figure}
\subsection{How does it work?}
\Large \raggedright IVUS is an invasive procedure and, as such, comes with the risks associated with any invasive procedure.

The entire procedure might take less than an hour or as long as several hours:

An area around the groin will be shaved and cleaned in preparation for the insertion of a catheter (a thin tube).
 1. A mild sedative is administered to aid in relaxation.

2. A local anesthetic is injected into the catheter site.

3. The imaging physician directs this catheter, painlessly, through arteries until it reaches the area to be studied.

4. A guide wire with an ultrasound probe on its tip is inserted into the catheter and guided to the furthest position to be imaged.

5. Sound waves are emitted from the probe. The probe also receives and returns the echo information, sending images to a computer.

6. The guide wire is held in place and the probe is slid backwards - usually under steady, smooth, motorized control - sending and receiving ultrasound images along the way.

7. After the catheter is reomoved, patients must lie flat for two to six hours.

8. If a patient lives more than an hour's drive from the hospital, they may need to stay overnight. Due to the administration of a mild sedative, patients will not be allowed to drive themselves home. If a patient arrives without a companion to take them home—and they are not staying overnight—the procedure will be cancelled and rescheduled.
\subsection{Results of the Intravascular Ultrasound}
\raggedright The blood vessel wall inner lining, atheromatous disease within the wall, and connective tissues are echogenic (they return echoes making them visible on the monitor). Blood and healthy muscular tissue are echolucent (they return no images, just black spaces on the monitor).

Heavy calcium deposits are very echogenic, which means they reflect sound, and are distinguishable by shadowing. Heavy calcifications are reflected as bright images with shadowing behind it.

Patients need to meet with their physician to discuss their test results and any recommended treatment plans.

\setlength{\parskip}{5em}
            \centering
                ********

 \setlength{\parskip}{1em}
\huge \raggedright References-



\Large wikipedia,Cedars Sinai,britannica,labcompare.

\setlength{\parskip}{6em}
            \centering
                ********

\end{document}
