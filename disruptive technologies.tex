\documentclass[11pt]{article}
\title{\Huge \textbf{ASSIGNMENT}}
\usepackage{graphicx}
\graphicspath{{image/}}
\begin{document}
\maketitle
\centering On

\Huge "Disruptive innovations in Healthcare "
\setlength{\parskip}{0.5em}

\emph{\large Submitted by:}
\large Priyanka Harde

\large Roll No.:21111039
\begin{figure}[h]
\begin{center}
\includegraphics[scale=0.7]{nit.jpg}
\end{center}
\end{figure}

\textsc{\Large NATIONAL INSTITUTE OF TECHNOLOGY}

\large Under the supervision of : SAURABH GUPTA 
\clearpage 
\tableofcontents
\clearpage
\section{\huge  ACKNOWLEDGEMENT}
\Large \raggedright I would like to express my sincere thanks and gratitude to Saurabh Sir for letting me work on this project. I am very grateful to him for his support and guidance in completing this project.

I am thankful to my parents as well. I was able to successfully complete this project with the help of their guidance and support. Finally, I want to thank all my dear friends as well..





\section{\textbf{\huge INTRODUCTION}}
\raggedright Disruptive innovations in healthcare can influence a new system that provides a continuum of care focused on each individual patient's needs, rather than focusing primarily on complex disorders and urgent health crises. Because of advances in diagnostic and therapeutic technologies, NPs and physician assistants can competently diagnose and treat disorders that would have previously required a physician.

\section{\textbf{\huge Technologies}}
\begin{figure}
\centering \includegraphics[scale=1]{disruptive.png}
\end{figure}
\raggedright 1. IoT

Our network of connected devices, combined with the internet and the cloud, allows for an astronomical exchange of data to take place. Transferring this data is very easy and convenient. With the addition of big data and AI, automation can be added to assist in the operation and regulating equipment.

2. EHRs

Electronic health records have gotten a facelift over the years. With the IoT, big data and devices’ connectivity provide up-to-date information about a patient at their point of care. In our age of technology, the way we handle patient data is changing on all fronts.

3. Remote Care

Remote patient care is solving many problems in healthcare. Relying on the convenience that the IoT provides in transferring data between devices, remote care offers convenience while maintaining quality care for patients. Remote patient monitoring and telehealth are made possible through video conferencing technology, big data, and wearable technology.

4. 3D Printing

3D printing impacts many industries in reducing labor costs while increasing production rates, and healthcare is one industry tapping into its enormous potential. Although initial prices may be high, 3D printing technology is developing rapidly every day, reducing the cost of manufacturing prototypes, prosthetics, tissue and skin, and even pharmaceuticals.

5. LASIK

Advancements in laser technology have made it easy for physicians and affordable for patients, eliminating their reliance on eyeglasses and contacts and electing for a more permanent vision-correction process. 

6. Retail Clinics

Retail health clinics, or a nurse-in-a-box, are essentially walk-in clinics found in retail stores, supermarkets, and pharmacies — such as a CVS store. Retail clinics are disrupting healthcare by providing convenient, quality care to patients for minor illnesses such as allergies, cold and flu, and minor burns and sprains. An essential part that makes retail clinics possible is EHRs.

7. Augmented Reality

Augmented reality is yet another emerging technology disrupting the healthcare industry—augmented reality supplements reality with images and sounds to create its own type of extended reality. Gaming, retail, and education are taking advantage of this new technology, and healthcare is capitalizing on augmented reality for its educational benefits.

8. Precision Medicine

Of course, every healthcare professional knows that not every patient is the same, and in specific instances, shouldn’t be treated medically as such. Precision medicine provides disease treatment and preventative measures based on an individual’s environment, lifestyle, and genetic makeup. This method moves away from the blanket, all-purpose approach to treating cystic fibrosis and cancer diseases.

9. Blockchain

One primary concern with the exchange of sensitive and proprietary data is secure. Whether it pertains to patients or professionals, healthcare data could prove to be catastrophic if it falls into the wrong hands. Blockchain can be used to secure any patient data, allowing for easy transferring while at the same time keeping it secure — only allowing those with consent to access it.
\section{\huge Conclusion}
\raggedright While past performance is not a guarantee of future results, it’s almost a sure bet that the process of disruptive innovation will improve healthcare as it has for so many other industries. As new products meet consumer-driven demand, health and economic gains will spread to stakeholders and beyond.

Thankfully, in the midst of disruption existing digital health services are already finding ways to deliver simplified, cost-effective care on demand. However, they can’t transform the industry alone. Smart policy, private enterprise, and continued support of entrepreneurs will be necessary for society to reap the benefits of consumer-focused healthcare in the years to come.



 
\Large References-
journals.www.com ,getreferralmd.com,healthio.milliman.com
\end{document}