\documentclass[11pt]{article}
\title{\Huge \textbf{ASSIGNMENT}}
\usepackage{graphicx}

\graphicspath{{image/}}

\begin{document}
\maketitle
\begin{center}
\huge On 
\end{center} 

\huge\centering 5 Solutions to Covid19 provided by Biomedical Engineers​

\setlength{\parskip}{0.5em}\emph{\large Submitted by: }
\large Priyanka Harde

\large Roll No.:21111039
\begin{figure}[h]
\begin{center}
\includegraphics[scale=0.5]{nit.jpg}
\end{center}

\end{figure}



\textsc{\large NATIONAL INSTITUTE OF TECHNOLOGY ,RAIPUR}
\setlength{\parskip}{0.5em}




\textit{\Large Under the Supervision of:}
\textbf{\Large Saurabh Gupta}
\clearpage
\tableofcontents
\clearpage

\raggedright
\section{\Huge \raggedright ACKNOWLEDGEMENT}
\Large I would like to express my sincere thanks and gratitude to Saurabh Sir for letting me work on this project. I am very grateful to him for his support and guidance in completing this project.

I am thankful to my parents as well. I was able to successfully complete this project with the help of their guidance and support. Finally, I want to thank all my dear friends as well..
\section{\textbf{\Huge INTRODUCTION}}
The role of a Biomedical Engineer includes designing biomedical equipment and devices to aid the recovery or improve the health of individuals. This can include internal devices, such as stents or artificial organs, or external devices, such as braces and supports (orthotics).

\section{\huge 5 Solutions to Covid19 ​}

1)Continuous Positive Airway Pressure (CPAP)-

The next step up in treating COVID-19 patients is Continuous Positive Airway Pressure (CPAP) which is initially intended to prevent airways collapse in sleep apnoea patients, but has been shown to be beneficial to COVID patients if applied early enough in the progression of the disease.

2)Ventilators-

Patients who cannot breathe spontaneously need to be put on a ventilator. Ventilators are capable of replacing the breath function and patients in an advanced state of respiratory distress are usually intubated and sedated at the beginning of the treatment.

3)Wearable tech and early illness detection-

During the pandemic, researchers have taken full advantage of the proliferation of smartwatches, smart rings and other wearable health and wellness technology. These devices can measure a person’s temperature, heart rate, level of activity and other biometrics. With this information, researchers have been able to track and detect COVID-19 infections even before people notice they have any symptoms.

4)Rapid case identification-

Rapid diagnostic test (RDT) of a sample of the respiratory tract of a person helps to detect the viral proteins (antigens) related to COVID-19 virus. This ensures a speedy and accurate diagnosis and its usage is CDC-approved

5)PPE(personal protective equipment)-
 
(a)  protective clothing- Personal protective devices such as protective clothing was designed and made using surgical drapes and plastics  

(b) testing booth- Diagnostic testing booths equipped with HEPA filters  were made to reduce the use of   surgical gowns and medical supplies. This booth is more comfortable for healthcare personnel and lowers the probability of infection during the removal of gowns

(c) face shield- Lots of innovative solutions for face shields were developed . These range from face shields made using 3D-printed frames and thin plastic binder sleeves, to fashionable consumer face shields.

\section{\Huge Conclusion}

The COVID-19 pandemic is a disruptive event in our history. It is a “reset” on the way we live and do things.  Even though the current pandemic has had a negative impact on the world, it has given our scientists, engineers and innovators a challenge and motivation to create solutions to the problem.  Humans survive because of their creativity, innovations, and ability to create new solutions.







References-www.bcu.ac.uk,www.clinicallabmanager.com,
www.nature.com,www.narayanahealth.org,starfishmedical.com
\end{document}